%===================================================================================
% Chapter: Introduction
%===================================================================================
\chapter*{Introducción}\label{chapter:introduction}
\addcontentsline{toc}{chapter}{Introducción}
%===================================================================================

La creatividad se ha vuelto la consigna del mundo moderno: toda persona e institución
está motivada a concebir, innovar y pensar de una forma diferente. 
Esto hace un llamado a diseñar una nueva clase de tecnología, orientada a
\textit{asistir} a los seres humanos en tareas de índole creativa.

Generación Procedural de Contenido (PCG por sus siglas en inglés)
es, hoy en día, un término que pudiera ser utilizado para describir este campo de imaginación
computacional. PCG se define como la creación de contenido virtual con interacción
limitada o indirecta del usuario \cite{bib:5}. En este contexto, \textit{contenido} habla
de elementos encontrados habitualmente en videojuegos: niveles, mapas,
reglas de juego, texturas, historias, objetos, misiones, música,
armas, vehículos, ciudades, etc. En otras palabras, PCG se refiere
a programas que pueden \textit{crear} contenido por sí mismos
o en conjunto con seres humanos. 

La razón principal por la que PCG se ha ganado un lugar en la actualidad
es la reducción de costos de producción de dicho contenido \cite{bib:11}. Las tareas en las que se
requiere de productos creativos generalmente necesitan
una gran cantidad de tiempo invertidoAdemás, la contratación del profesionales
tiene un costo superior respecto a la sustitución del trabajo de los humanos por un algoritmo.
Es común que grupos de cientos
de profesionales trabajando en equipo demoren más de un año en terminar un videojuego.

La Generación Procedural de Contenido a través de Aprendizaje de Máquina (PCGML por sus siglas en inglés)
es el término acuñado en la actualidad para la estrategia de controlar
generadores de contenido mediante ejemplos. Varios sistemas
de PCGML se basan en modelos estadísticos entrenados con ejemplos
del contenido que se desea generar \cite{bib:1}. 

La diferencia entre PCGML y los demás acercamentos a 
generar contenido es en la utilización de los modelos
de aprendizaje de máquina.
En PCGML, el contenido es generado \textit{directamente} por el modelo.
En contraste, el resto de las vertientes de PCG en las que se usa
aprendizaje de máquinas, utilizan modelos
entrenados para \textit{evaluar} el contenido, la generación
ocurre \textit{buscando} en el espacio de contenidos.

Los ejemplos entrenantes que se proveen
a los sistemas deben ser diseñados de antemano por artistas, lo cual
conlleva un trabajo manual exhaustivo. Si a esto se suma que el número
de ejemplos necesarios para que los algoritmos de aprendizaje de máquina
converjan es muy grande, la tarea se vuelve prácticamente infactible.
Es frecuente que el esfuerzo de producir ejemplos de alta calidad
tenga un costo tan alto que los artistas prefieran idear
el contenido final ellos mismos. Esto representa la principal
\textit{tensión} de PCGML: se necesitan muchos ejemplos de alta calidad
para generar contenido creativo exitoso.

La generación de contenido, incluida dentro de herramientas de diseño,
puede aumentar la creatividad de artistas individuales \cite{bib:4}.
Dicho contenido se puede ajustar al usuario
\cite{bib:7}, utilizando aprendizaje profundo para modelar la respuesta
del usuario a un tipo de contenido en específico tratando de maximizar
el disfrute.

\section*{Wave Function Collapse}

\textit{Wave Function Collapse} (WFC) \cite{bib:16}
es un algoritmo de generación de texturas de tamaño arbitrario
a partir de \textit{una} textura de ejemplo.

Este algoritmo ha sido utilizado de manera comercial en juegos
como Caves of Qud \cite{bib:17} y Bad North \cite{bib:18}. Karth y Smith 
publicaron un artículo en el que describen el algoritmo en profundidad,
discuten cuánto ha influido en la generación de texturas y lo
contextualizan en términos de satisfacción de restricciones \cite{bib:2}.
Lefebvre, Wei y Barnes han experimentado con esconder mensajes
usando WFC, haciendo énfasis en el hecho de que las imágenes
con información escondida no son distinguibles de las imágenes
generadas arbitrariamente \cite{bib:14}. Scurti y Verbrugge
publicaron un artículo mostrando cómo se puede utilizar
WFC para generar caminos a seguir por NPC (\textit{non-playable characters}),
a partir de un simple boceto y cómo integrar el camino generado a
un juego determinado.

En WFC se aprecian tres funciones: un clasificador de patrones 
que dado un ejemplo, lo transforma en un conjunto de identificadores;
una función de validez de adyacencia, la cual dados dos patrones determina
si su superposición cumple con todas las restricciones; un 
renderizador de patrones que dado un patrón decide qué contenido generar.

Al mezclar WFC con aprendizaje por discriminación \cite{bib:12}
creando una herramienta de iniciativa mixta 
para generar contenido \cite{bib:4},
se obtienen resultados
alentadores \cite{bib:3}. En el aprendizaje por 
discriminación, el modelo aprende a juzgar si un ejemplo nuevo
puede ser contenido válido o deseable, pero dicho modelo no aprende
a generar candidatos.

(Falta añadir las limitaciones de WFC)

En este trabajo se propone diseñar y validar un nuevo enfoque de generación
automática de contenido, basado en WFC, que permita relajar las restricciones 
aprendidas de la entrada para alcanzar un mayor nivel de generalización.

\subsection*{Objetivos}

\section*{Organización de la Tesis}

