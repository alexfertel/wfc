%===================================================================================
% Chapter: Introduction
%===================================================================================
\chapter*{Introducción}\label{chapter:introduction}
\addcontentsline{toc}{chapter}{Introducción}
%===================================================================================

% \subsection*{Objetivos}

% \subsection*{Contribuciones}

% \section*{Organización de la Tesis}

La creatividad se ha vuelto la consigna del mundo moderno: toda persona e institución
está motivada a concebir, innovar y pensar de una forma diferente. 
Esto hace un llamado a diseñar una nueva clase de tecnología, orientada a
\textit{asistir} a los seres humanos en tareas de índole creativa.

Generación Procedural de Contenido (PCG por sus siglas en inglés)
es, hoy en día, un término que pudiera ser utilizado para describir este campo de imaginación
computacional. PCG se define como la creación de contenido virtual con interacción
limitada o indirecta del usuario \cite{bib:5}. Cuando se dice \textit{contenido} se habla
de elementos encontrados habitualmente en videojuegos: niveles, mapas,
reglas de juego, texturas, historias, objetos, misiones, música,
armas, vehículos, ciudades, etc. En otras palabras, PCG se refiere
a programas que pueden \textit{crear} contenido por sí mismos
o en conjunto con seres humanos. 

La razón principal por la que la PCG se ha ganado un lugar en la actualidad
es la reducción de costos de producción de dicho contenido. Las tareas en las que se
requiere de productos creativos generalmente necesitan
una gran cantidad de tiempo invertido, además,
contratar humanos es caro con respecto a tener un algoritmo
que sustituya su trabajo. Es común que grupos de cientos
de profesionales trabajando en equipo demoren más de un año en terminar un videojuego.

La Generación Procedural de Contenido a través de Aprendizaje de Máquina (PCGML por sus siglas en inglés)
es el término acuñado en la actualidad para la estrategia de controlar
generadores de contenido mediante ejemplos. Varios sistemas
de PCGML se basan en modelos estadísticos entrenados en ejemplos
del contenido que se desea generar \cite{bib:1}. 

Hay dos vertientes principales dentro de PCG: 
los métodos basados en búsqueda \cite{bib:8} y los métodos
basados en \textit{solvers} \cite{bib:9} \cite{bib:10}.
Mientras que estos métodos \textit{pudieran} utilizar modelos
entrenados para \textit{evaluar} el contenido, la generación
ocurre \textit{buscando} en el espacio de contenidos.
En contraste, en PCGML, el contenido es generado 
\textit{directamente} por el modelo.

Los ejemplos entrenantes que se proveen
a los sistemas deben ser diseñados de antemano por artistas, lo cual
conlleva un trabajo manual exhaustivo. Si a esto se suma que el número
de ejemplos necesarios para que los algoritmos de aprendizaje de máquina
converjan es muy grande, la tarea se vuelve prácticamente infactible.
Es frecuente que el esfuerzo de producir ejemplos de alta calidad
sea lo suficientemente caro para que los artistas prefieran idear
el contenido final ellos mismos. Esto representa la principal
\textit{tensión} de PCGML: se necesitan muchos ejemplos de alta calidad
para generar contenido creativo exitoso.

La generación de contenido, incluida dentro de herramientas de diseño,
puede aumentar la creatividad de artistas individuales \cite{bib:4}.
Más emocionante aún, dicho contenido se puede ajustar al usuario
\cite{bib:7}, utilizando aprendizaje profundo para modelar la respuesta
del usuario a contenido específico tratando de maximizar
el disfrute.

\textit{Wave Function Collapse} (WFC) \cite{bib:2} es un algoritmo generativo probado
comercialmente que resulta muy atractivo, pues
solo necesita un ejemplo (dato entrenante) para \textit{aprender} el modelo.
Esto muestra un primer acercamiento a cuán
prometedor es WFC para resolver la tensión planteada anteriormente.
(Ejemplos, ejemplos, ejemplos.)

En dicho sistema se aprecian tres funciones
cuya naturaleza es atractiva a explorar: un clasificador de patrones 
que dado un ejemplo, lo transforma en un conjunto de identificadores,
una función de validez de adyacencia, la cual dados dos patrones determina
si su superposición cumple con todas las restricciones y un 
renderizador de patrones que dado un patrón decide qué contenido generar.

Al unir WFC con aprendizaje por discriminación se obtienen resultados
alentadores \cite{bib:3}, creando una herramienta de iniciativa mixta 
para generar contenido \cite{bib:4}. En el aprendizaje por 
discriminación, el modelo aprende a juzgar si un artefacto candidato
puede ser contenido válido o deseable, pero dicho modelo no aprende
a generar candidatos. (MGG and LGG)

En este trabajo se propone diseñar y validar una propuesta de generación
automática de contenido, basada en WFC, que permita relajar las restricciones 
aprendidas de la entrada para alcanzar un mayor nivel de generalización.



