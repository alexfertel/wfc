%===================================================================================
% Chapter: Background
%===================================================================================
\chapter{Generación Procedural de Contenido}\label{chapter:background}
% \addcontentsline{toc}{chapter}{Generación Procedural}
%===================================================================================

\section{Shallow Overview}
\section{Síntesis de modelos}
\subsection{Síntesis de modelos discretos}

\subsection{NP completitud}

\section{Wave Function Collapse}
his program generates bitmaps that are locally similar to the input bitmap.
\textit{WaveFunctionCollapse} es un algoritmo de generación de \textit{bitmaps}
que son localmente similares a un \textit{bitmap} de entrada.

Similaridad local significa que: 
\begin{enumerate}
    \item Cada submatriz de tamaño NxN en la salida ocurre al menos una vez
        en la matriz de entrada.
    \item (Débil) La distribución de las submatrices de tamaño NxN en la entrada debe ser
        similar a la distribución de las submatrices de tamaño NxN en la salida dado
        una cantidad de matrices de salida lo suficientemente grande. Es decir, la probabilidad
        de encontrar una submatriz particular en la salida debe estar cercana a la densidad
        de dicha submatriz en la entrada. 
\end{enumerate}


\subsection{Problemas de Satisfacción de Restricciones}
\subsection{Clasificación, validación y renderizado}

El algoritmo consta de tres funciones principales:

\begin{itemize}
    \item Una función de clasificación que...
    \item Una función de validación que...
    \item Una función de renderizado que...
\end{itemize}

\subsection{Modelos de WFC}

Maxim Gumin implementó 2 modelos de adyacencia diferentes:
El \textit{Simple Tiled Model} es más sencillo y menos costoso
computacionalmente que el \textit{Overlapping Model}, sin embargo
ambos modelos de adyacencia muestran exactamente la misma estructura
en el resto de la implementación, porlo que se puede apreciar
que desde su concebido fue atractivo modificar dichas funciones
para extender el algoritmo.

\subsubsection{Tiled}
\subsubsection{Overlapping}

\subsection{Aprendizaje por discriminación}


\subsubsection{MGG and LGG}

\subsection{Trabajos relacionados}

WFC ha sido utilizado de manera comercial en juegos
como Caves of Qud \cite{bib:17} y Bad North \cite{bib:18}.
Karth y Smith publicaron un artículo en el que describen 
el algoritmo en profundidad,
discuten cuánto ha influido en la generación de texturas y lo
contextualizan en términos de satisfacción de restricciones \cite{bib:2}.
Lefebvre, Wei y Barnes han experimentado con esconder mensajes
usando WFC, haciendo énfasis en el hecho de que las imágenes
con información escondida no son distinguibles de las imágenes
generadas arbitrariamente \cite{bib:14}. Scurti y Verbrugge
publicaron un artículo mostrando cómo se puede utilizar
WFC para generar caminos a seguir por NPC (\textit{non-playable characters}),
a partir de un simple boceto y cómo integrar el camino generado a
un juego determinado.


\section{Discusión}

