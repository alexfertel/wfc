
\documentclass[a4paper, 10pt]{article}
\usepackage[spanish]{babel}
\usepackage[utf8]{inputenc}
\usepackage[top=1 in, left=1 in, right=1 in, bot=1 in]{geometry}
\usepackage{amsmath, amsthm, amsfonts}
\usepackage{graphics}
\usepackage{float}
\usepackage{epsfig}
\usepackage{amssymb}
\usepackage{latexsym}
\usepackage{newlfont}
\usepackage{epstopdf}
\usepackage{amsthm}
\usepackage{epsfig}
\usepackage{caption}
\usepackage{multirow}
\usepackage[colorlinks]{hyperref}
\usepackage[x11names,table]{xcolor}
\usepackage{graphics}
\usepackage{wrapfig}
\usepackage[rflt]{floatflt}
\usepackage{multicol}
\usepackage{longtable,multirow,booktabs}
\usepackage{listings} \lstset {language = Python, basicstyle=\bfseries\ttfamily, keywordstyle = \color{blue}, commentstyle = \bf\color{green}, backgroundcolor = \color{gray!5}, stringstyle = \color{yellow!5}}
\usepackage{titlesec}

% \titleformat{\section}{\filcenter}{}{8pt}{}

% \setcounter{secnumdepth}{}
\renewcommand{\thesection}{\Roman{section}}

\title{WaveFunctionCollapse, un enfoque diferente}
\author{Alexander Antonio González Fertel C-512 \hfill
		\href{mailto:a.fertel@estudiantes.matcom.uh.cu}{a.fertel@estudiantes.matcom.uh.cu}\\
		Joel David Hernández Cruz C-511 \hfill
		\href{mailto:j.cruz@estudiantes.matcom.uh.cu}{j.cruz@estudiantes.matcom.uh.cu}\\
		Hieu Do Ngoc C-511 \hfill}
\date{}

\begin{document}
	\maketitle
	% \newpage

	\section{Introducción}
	Generación Procedural de Contenido via Aprendizaje de Máquina (PCGML por sus siglas en inglés)
	es el término acuñado en la actualidad para la estrategia de controlar
	generadores de contenido mediante ejemplos \cite{bib:1}. Existen acercamientos
	donde se utiliza PCGML que entrenan modelos estadísticos basados en ejemplos
	del contenido deseado, provistos de antemano por artistas, sin embargo,
	muchas veces el aprendizaje de máquina funciona mejor mientras más
	datos entrenantes hayan, pero el esfuerzo de producir datos entrenantes
	de calidad es frecuentemente lo suficientemente caro como para
	que los artistas estén mejor haciendo el contenido ellos mismos.
	
	La unión de un generador de contenido probado comercialmente,
	WaveFunctionCollapse (WFC) \cite{bib:2}, con el aprendizaje por discriminación
	muestra resultados muy atractivos \cite{bib:3} para las herramientas
	de generación de iniciativa mixta \cite{bib:4}. En el aprendizaje por 
	discriminación, el modelo aprende a juzgar si un artefacto candidato
	puede ser contenido válido o deseable, pero dicho modelo no aprende
	a generar candidatos. WFC es un sistema de generación de contenido
	guiado por ejemplos, donde el contenido generado cumple propiedades
	locales de la entrada. Esto se logra resolviendo un Problema de Satisfacción
	de Restricciones, en el cual las restricciones están implícitas. 
	
	En dicho sistema se aprecian tres funciones
	cuya naturaleza es atractiva a explorar: un clasificador de patrones 
	que dado un ejemplo, lo transforma en un conjunto de identificadores,
	una función de validez de adyacencia, la cual dados dos patrones determina
	si su superposición cumple con todas las restricciones y un 
	renderizador de patrones que dado un patrón decide qué forma tiene
	el contenido que genera ese patrón.
	
	En la herramienta presentada por Karth y Smith \cite{bib:3} se muestra
	cómo el diálogo entre un diseñador y el algoritmo permite mejorar iterativamente
	el contenido generado, haciendo este más cercano al producto final
	deseado hasta que el diseñador se siente cómodo y confía en el algoritmo.
	Se propone utilizar un diálogo similar para la generación de música,
	en la cual ejemplos positivos y negativos moldean la salida del programa.

	\subsection{Formalización del Problema}
	Extrapolar la herramienta de iniciativa mixta a otro modelo distinto
	a la generación de texturas, donde se muestre que WFC es agnóstico y
	cómo puede ser utilizado en contextos afines a la reducción de costos.
	% Sea $I$ el conjunto de todas las imágenes RGB de tamaño arbitrario posibles,
	% $S$ el conjunto de todos los sonidos posibles en el rango audible humano,
	% $S^n$ todas las secuencias de $n$ sonidos y
	% $F:I \rightarrow S^n$ una función que va de las imágenes a las secuencias de
	% sonidos de tamaño $n$, se quiere
	% hallar una función que dadas $k$ (donde $k$ es pequeño, a lo sumo 30) 
	% \textit{imágenes con contexto} (ejemplos positivos o negativos)
	% se obtengan secuencias de sonido que mantengan propiedades \textit{locales} ($i$)
	% y \textit{coherentes} ($ii$) con respecto la secuencia de sonido que se obtiene de
	% evaluar las imágenes con contexto directamente en $F$.
	% Estas secuencias de sonido resultantes deben ser fácilmente
	% modificables en base a la adición de nuevas imágenes contextuales.
	
	\begin{itemize}
		\item[$i$.] Se refiere a que cada patrón \textit{positivo} de $NxN$ en la imagen,
			mapeado por $F$ a una secuencia de sonidos, debe aparecer en la secuencia final
			al menos 1 vez.
		\item[$ii$.] La distribución de los $NxN$ patrones en la imagen de entrada debe 
			ser similar a la distribución de $NxN$ patrones en un número suficientemente largo
			de secuencias de sonido generadas. Es decir, la probabilidad de encontrar
			una secuencia de sonidos en particular debe ser cercana a la densidad de dichos
			patrones en el contenido generado.  
	\end{itemize}
	
	\section{Propuesta de Solución}
	Se presenta como propuesta de solución utilizar un enfoque de aprendizaje por discriminación,
	en donde una modificación del algoritmo WFC logra los resultados deseados: una herramienta
	para artistas en la que se mantiene un diálogo entre usuario y programa hasta alcanzar
	un producto (generador de contenido) satisfactorio. Este producto recibe dos conjuntos de
	imágenes contextuales (positivo y negativo) a medida que transcurre el diálogo y en cada iteración
	genera sonidos a manera de \textit{portfolio}.
	(nótese que con artista se hace referencia a creadores de contenido, en este
	caso en particular, un diseñador junto a un músico).

	\subsection{WaveFunctionCollapse}
	WaveFunctionCollapse\footnote{https://github.com/mxgmn/WaveFunctionCollapse}
	es básicamente un solucionador de 
	\textit{Problemas de Satisfacción de Restricciones}\cite{bib:2} (CSP por sus siglas en inglés).
	El algoritmo recibe como entrada un ejemplo de contenido a generar, extrae patrones (subestructuras)
	de dicho ejemplo y aprende las relaciones entre dichos patrones, creando un conjunto
	de restricciones que mantienen las propiedades ($i$) y ($ii$), luego se genera
	un nuevo ejemplo de contenido resolviendo el CSP resultante y finalmente, se utiliza la
	asignación representada en la solución de dicho CSP como salida del algoritmo.

	\subsubsection{Clasificador de Patrones}
	La primera función importante que se encuentra al ejecutar el algoritmo es
	el \textit{clasificador de patrones}. Dicha función toma como entrada una matriz,
	de la cual extrae todas las submatrices de tamaño $N$, que es un parámtro del
	algoritmo, y devuelve indices para cada una de las submatrices, respetando unicidad. 
	La frecuencia con que ocurre cada patrón en el ejemplo es almacenada
	para utilizarla como función de distribución, al escoger el patrón que va a ser escogido
	en el dominio de una variable del CSP a resolver, además de utilizarse para el cómputo
	de la entropía de Shannon para cada una de las variables.
	Sea $\mathbb{M}$ el espacio de las matrices de enteros y sea $C$ el clasificador de 
	patrones, se tiene que $C:\mathbb{M} \rightarrow \mathbb{N}$. 

	\subsubsection{Validación de Adyacencia}
	Esta función recibe dos patrones y retorna si dichos patrones pueden aparecer
	adyacentes. Aquí es donde se hacen cumplir las restricciones a la hora de
	generar contenido, de esta forma se "propaga" la consistencia de los arcos
	en el CSP, dado que WFC implementa implícitamente AC3\cite{bib:2}.
	Sea $V$ esta función, se tiene que $V:\mathbb{N}\times\mathbb{N} \rightarrow \{0, 1\}$.

	\subsubsection{Renderizador de Patrones}
	Esta última función toma un patrón y da su representación en el contenido generado,
	es decir, cumple con la representación que tiene dicho patrón en el ejemplo.
	Sea $R$ el renderizador, se tiene $R:\mathbb{N} \rightarrow \mathbb{M}$.

	\begin{figure}[h]
		\begin{center}
			\includegraphics[width=1\textwidth]{images/fig1.png}			
		\end{center}
		\caption{\textit{Funciones en WFC}. En esta figura \textbf{f()} es el clasificador
		de patrones y \textbf{g()} es el renderizador de patrones. Se puede apreciar en la figura
		WFC aplicado a generación de texturas como caso particular en el que la entrada y la
		salida son texturas.}
		\label{fig:1}	
	\end{figure}	

	Como se puede apreciar en la Figura~\ref{fig:1}, estas tres funciones forman una especie
	de \textit{pipeline} con el cual podemos explorar distintos acercamientos, como se verá en
	la sección \ref{marco experimental}.

	\subsubsection{Generalización Más General}
	El algoritmo original asume que los patrones que se solapan en la imagen de ejemplo
	representan todas las adyacencias válidas, es decir, las restricciones que se aprenden
	sirven para mostrar el conjunto de adyacencias que pueden existir en el contenido
	generado, incluso si dos patrones no eran adyacentes en el ejemplo, estos pueden
	aparecer adyacentes en la salida. Karth y Smith acuñan este acercamiento como
	\textit{Generalización Más General}\cite{bib:3}. 
	
	Llámemos \textit{ejemplo contextual} a un ejemplo (imagen en el algoritmo original) 
	en el que se define que las restricciones obtenidas de los patrones de dicho ejemplo
	son positivas o negativas. Una restricción positiva es aquella que permite que
	dos patrones sean adyacentes, es decir, relaja el conjunto de restricciones, una
	restricción negativa es aquella que determina que dos patrones no pueden 
	encontrarse adyacentes en el contenido generado. Al añadir la posibilidad de añadir
	nuevos ejemplos contextuales manteniendo las relaciones ya aprendidas por el modelo descrito por WFC,
	se corrije el comportamiento anterior del generador, es decir, se actualizan las
	restricciones del modelo, lo cual
	logra un nivel de precisión alto en la búsqueda del espacio de contenido a generar deseado.
	Si un ejemplo es positivo, todas las relaciones de adyacencia entre los patrones
	delejemplo son positivos, análogamente para los ejemplos negativos.

	\subsubsection{\textit{Rationale}}
	El racional detrás de escoger este algoritmo está dado por las siguientes razones:
	\begin{itemize}
		\item Es agnóstico al contenido, expresar restricciones y obtener patrones son conceptos
			muy abstractos que se capturan bastante bien, dado que la entrada y la salida del algoritmo
			son matrices, alguien que use WFC sólo necesita hallar una representación en
			matrices del contenido a generar y como mapear entre las representaciones.
		\item Con un solo ejemplo se puede generar contenido de tamao arbitrario, lo que
			lo hace mucho mejor que la mayor parte de los modelos actuales que necesitan
			de grandes cantidades de datos.
		\item Modificarlo para poder usar aprendizaje por discriminación es bastante sencillo.
		\item Mantiene precisamente restricciones locales y débilmente las globales.
	\end{itemize}

	Se ilustra mediante la generación de sonido a través de una imagen, cuán 
	versátil es WFC.

	\section{Marco Experimental}\label{marco experimental}
	Se implementó un prototipo computacional en \textbf{python3} del algoritmo WFC,
	donde se extiende el algoritmo para recibir varios ejemplos. Dicho prototipo
	recibe una imagen, la lleva a una representación matricial, donde cada posición
	es un color y genera una imagen de tamaño arbitrario, definido con anterioridad
	para conformar con la representación que se necesita en la segunda parte del prototipo.
	Esta segunda parte convierte la imagen generada a una secuencia de sonidos de piano.
	Luego de la primera imagen de ejemplo, se permite la adición de imágenes contextuales
	para la modificación de la secuencia generada.

	\section{Resultados}
	Se pueden tomar dos acercamientos distintos a la hora de evaluar los resultados
	de un algoritmo de generación de contenido de este tipo.

	\subsection{Top-down. Rango de Expresividad}
	El rango de expresividad se refiere al espacio de contenido potencialmente
	deseable que el algoritmo es capaz de crear, incluyendo cuán sesgado está
	de crear un tipo de contenido en particular de ese espacio\cite{bib:5}.
	Esta evaluacion es realizada escogiendo métricas a través de las cuáles el contenido
	puede ser evaluado, para luego utilizar esas métricas como ejes para definir
	el espacio de contenido posible. Un grande número de piezas de contenido son
	entonces generadas y evaluadas con respecto a esas métricas y graficadas
	en un mapa de calor. Este mapa puede revelar sesgos en el generador y comparaciones
	de los mapas de calor de distintos parámetros del generador muéstran cuánto se
	puede controlar el generador.  

	Este acercamiento no es muy útil para la iniciativa mixta que hemos mostrado,
	puesto que la métrica que evalúa estos generadores es el nivel de satisfacción
	que muestre el artista interactuando con la herramienta.
	
	\subsection{Bottom-up. }
	Complementario al acercamiento top-down para la evaluación de los generadores
	de contenido, estudios cuantitativos sobre los usuarios del generador pueden 
	ser de gran beneficio para el aseguramiento de la calidad del contenido. El
	acercamiento más obvio sería preguntarle directamente a los artistas.

	Este acercamiento es mucho más factible en nuestro caso\cite{bib:4,bib:6},
	por lo que sería lo que usaríamos si tuvieramos una fuente de artistas dispuestos
	a realizar pruebas.
	

	\section{Conclusiones}
	Este trabajo propone una manera nueva de utilizar la versatilidad de WFC, extendiendo el
	trabajo de Karth y Smith, para mostrar una herramienta de iniciativa mixta para la generación
	de música.

	\subsection{Recomendaciones}
	Se recomienda explorar cómo influiría una aproximación de las tres funciones basado en algún
	acercamiento de aprendizaje de máquinas para mejorar dicha iniciativa mixta. Tal vez, aplicar
	aprendizaje por transferencia o buscar un intermedio entre Generalización Más General y 
	Generalización Menos General a la hora de aprender restricciones.


	\bibliography{paper}
	\bibliographystyle{ieeetr}


\end{document}
